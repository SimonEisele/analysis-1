%============================ MAIN DOCUMENT ================================
\PassOptionsToPackage{table}{xcolor}
\documentclass[
  a4paper,
  bibliography=totoc,
  listof=totoc,
  invert-title,
  titleimage-ratio=13
]{bfhpub}

\usepackage[french,english,main=ngerman]{babel}  % https://www.namsu.de/Extra/pakete/Babel.html
\LoadBFHModule{listings,terminal,boxes}

%---------------------------------------------------------------------------
% Documents paths
%---------------------------------------------------------------------------
\makeatletter
\def\input@path{{content/}}
\makeatother

%---------------------------------------------------------------------------
% Hyperref Package (Create links in a pdf)
%---------------------------------------------------------------------------
\usepackage[
	,bookmarks
	,plainpages=false
	,pdfpagelabels
  ,pdfusetitle
	,backref = {false}          % No index backreference
	,colorlinks = {true}        % Color links in a PDF
	,hypertexnames = {true}     % no failures "same page(i)"
	,bookmarksopen = {true}     % opens the bar on the left side
	,bookmarksopenlevel = {0}   % depth of opened bookmarks
	,linkcolor=.
	,filecolor=.
	,urlcolor=.
	,citecolor=.
]{hyperref}

%---------------------------------------------------------------------------
% Base packages
%---------------------------------------------------------------------------
% Include Packages
\usepackage{amsmath}          % various features to facilitate writing math formulas
\usepackage{amsthm}           % enhanced version of latex's newtheorem
\usepackage{amsfonts}         % set of miscellaneous TeX fonts that augment the standard CM
\usepackage{amssymb}          % mathematical special characters

\usepackage{siunitx}

\usepackage{graphicx}         % integration of images
\usepackage{float}            % floating objects

\usepackage{caption}          % for captions of figures and tables
\usepackage{subcaption}       % for subcaptions in subfigures
\usepackage{wrapfig}

\usepackage{exscale}          % mathematical size corresponds to textsize
\usepackage{multirow}         % multirow emables combining rows in tables
\usepackage{multicol}
\usepackage{makecell}

\usepackage{longtable}
\usepackage{adjustbox}

\usepackage[colorinlistoftodos]{todonotes}

\usepackage[table]{xcolor}

\usepackage{array}

\usepackage{mathabx}

\usepackage{attachfile2}

\usepackage{pgfplotstable}

\usepackage{totcount}         % Abbildungs- und Tabellenverzeichnis nur bei vorhandenen Einträgen drucken
\newtotcounter{figures}
\newtotcounter{tables}
\AtBeginEnvironment{figure}{\stepcounter{figures}}
\AtBeginEnvironment{table}{\stepcounter{tables}}

%---------------------------------------------------------------------------
% Graphics paths
%---------------------------------------------------------------------------
\graphicspath{{pictures/}{figures/}}

%---------------------------------------------------------------------------
% Blind text -> for dummy text
%---------------------------------------------------------------------------
\usepackage{blindtext}    
\usepackage{letltxmacro}   
\LetLtxMacro{\blindtextblindtext}{\blindtext}

\RenewDocumentCommand{\blindtext}{O{\value{blindtext}}}{
	\begingroup\color{BFH-Gray}\blindtextblindtext[#1]\endgroup
}

%---------------------------------------------------------------------------
% Bibliography Package
%---------------------------------------------------------------------------
\usepackage{csquotes}
\usepackage[backend=biber,style=ieee]{biblatex}
\addbibresource{references.bib}

%---------------------------------------------------------------------------
% Glossary Package
%---------------------------------------------------------------------------
\usepackage[nonumberlist, acronym]{glossaries-extra}
\setabbreviationstyle[acronym]{long-short}
\makeglossaries
\input{content/glossary.tex}

%---------------------------------------------------------------------------
% Makeindex Package
%---------------------------------------------------------------------------
\usepackage{makeidx}
\makeindex

\begin{document}

%------------ START FRONT PART ---------------------------------------------------------------------------------------------------------------------------
\frontmatter % Nummerierung der Seiten in römischen Zahlen

%----------------  BFH tile page   -----------------------------------------
  \title{Projekt 4: Kombinatorische Logik}
  \subtitle{BTE5213 Laborprojekte, Herbstsemester 2025/2026\\Protokoll}
  \author{Janis Aebischer \and Simon Eisele (Gruppe D)}
  \department{Technik und Informatik}
  \institute{Elektrotechnik und Informationstechnologie}
  \version{1.0}
  \titlegraphic{\includegraphics{Bild1.png}}
  \partnerlogo{\includegraphics[height=\height]{bfh-logo.pdf}}

  \maketitle

  \begin{description}
    \item[] Durchführung des Versuchs am 24. Oktober 2025
    \item[] Betreut durch Prof. Dr. Torsten Mähne
  \end{description}

%----------------  Abstract   ----------------------------------------------
\section*{Abstract}
\label{sec:abstract}
Ziel dieses Laborprojekts ist es, die im Modul Digitaltechnik erlernten Grundlagen zu vertiefen und praktisch anzuwenden. Dazu wurden mehrere kombinatorische Schaltungen
entwickelt, deren logische Funktionen mithilfe der Booleschen Algebra und KV-Diagrammen vereinfacht wurden. Die entwickelten Schaltungen wurden in Logisim-evolution umgesetzt,
simuliert und anschliessend auf dem Leguan-Board der BFH getestet. Durch die Verknüpfung von Theorie, Simulation und praktischer Umsetzung konnte das Verständnis für den Aufbau
und die Funktionsweise digitaler Logiksysteme gezielt erweitert werden. Bei den Logikfunktionen handelt es sich um folgende Baugruppen. BCD-zu-7-Segment-Decoder, Ripple-Carry-Addierer,
 Borrow-Bit-Subtrahierer, Ergebnisumschalter und Binär-zu-BCD-Converter.
\thispagestyle{plain}

%----------------  Table of contents   -------------------------------------
\clearpage
\tableofcontents

%------------ START MAIN PART ----------------------------------------------------------------------------------------------------------------------------
\mainmatter % Beginn mit normaler Nummerierung der Seiten

%----------------  Introduction   ------------------------------------------
\clearpage
\section{Einleitung}
\label{sec:introduction}
Ziel des Projekts 4 „Kombinatorische Logik“ war es, die Software \gls{le}~\cite{burchLogisimevolution2024}\footnote{\url{https://github.com/logisim-evolution/logisim-evolution}},
sowie das \gls{lb}~\cite{HomeLeguanDevelopers}\footnote{\url{https://leguan.ti.bfh.ch/}} der \gls{bfh} kennenzulernen. Dabei sollten die aus dem Modul BTE5021-Digital
\glqq Elektronik Grundlagen\grqq~gewonnenen Kenntnisse angewendet und gefestigt werden.

Hierfür wurden verschiedene Schaltungen entwickelt, in \gls{le} aufgebaut und zum Testen auf den \gls{fpga} des \gls{lb} geladen. Die zu entwickelnden Schaltungen wurden in der
Aufgabenstellung~\cite{tamselKombinatorischeLogikProjekt2025} definiert:
\begin{itemize}
    \item \gls{bcd}-zu-7-Segment-Decoder
    \item Ripple-Carry-Addierer
    \item Borrow-Bit-Subtrahierer
    \item Ergebnisumschalter
    \item Binär-zu-\gls{bcd}-Konverter
\end{itemize}

\noindent Für die Entwicklung der Schaltungen wurden jeweils Wahrheitstabellen erstellt. Daraus konnten die entsprechenden \gls{kv}-Diagramme abgeleitet und anschliessend die minimalen Funktionen in
disjunktiver (Minterme) oder konjunktiver (Maxterme) Form bestimmt werden. Beim \gls{bcd}-zu-7-Segment-Decoder mussten zusätzliche Bedingungen berücksichtigt werden. Für die Funktionen
durften jeweils nur bestimmte Grundgattertypen verwendet werden. Dies bot die Möglichkeit, das Umformen mittels der Booleschen Algebra anzuwenden und zu üben.

Vor der praktischen Durchführung des Versuchs wurden die Vorbereitungsaufgaben~\cite[p.~3]{tamselKombinatorischeLogikProjekt2025} bearbeitet. Dabei ging es darum, die Schaltungen
für den \gls{bcd}-zu-7-Segment-Decoder gemäss den gestellten Bedingungen zu entwickeln, sowie ein Konzept für die logische Schaltung zur Umwandlung einer 8 bit-Binärzahl in 3 \gls{bcd}-Ziffern
zu erstellen.

Der praktische Versuch wurde im Labor durchgeführt und dauerte vier Lektionen. Im Anschluss blieb noch eine Woche, um fehlende Schaltungen zu vervollständigen
und den technischen Bericht fertigzustellen.

Die Grundlagen zur Durchführung dieses Projekts bildet das Modul BTE5021-Digital \glqq Elektronik Grundlagen\grqq. Die wichtigsten Inhalte sind im Skript
Digitaltechnik~\cite{jacometDigitaltechnikGrundlagen2023} zusammengefasst.

%----------------  Methods   -----------------------------------------------
\clearpage
\section{Methoden und Materialien}
\label{sec:methods}
%----------------Versuchs- und Projektaufbau--------------------------------------------------------------------------------------------------------------------------------------------------------
\subsection{Versuchs- und Projektaufbau}
\subsubsection{Ziele}
Die Herangehensweise wurde durch die Aufgabenstellung~\cite{tamselKombinatorischeLogikProjekt2025} bereits ziemlich genau definiert. Als erstes mussten die 
Vorbereitungsaufgaben~\cite[p.~3]{tamselKombinatorischeLogikProjekt2025} gelöst werden, welche grundsätzlich aus drei Teilen bestehen:
\begin{itemize}
    \item Funktionen für den \gls{bcd}-zu-7-Segment-Decoder ermitteln
    \item Konzept für den Algorithmus zur Umwandlung einer 8 Bit-Binärzahl in 3 \gls{bcd}-Ziffern
    \item \gls{le} kennelernen und über die Möglichkeiten dieser Software informieren.
\end{itemize}
Diese Aufgabenstellungen wurden vor der Projektdurchführung abgearbeitet und bereits zur Bewertung abgegeben. Sie bilden zusammen die Grundlage für dieses Projekt.

Während des Versuchs vor Ort sollen die anderen Aufgaben gemäss der Aufgabenstellung~\cite{tamselKombinatorischeLogikProjekt2025} bearbeitet werden. Diese Aufgaben sind
der Aufgabenstellung zu entnehmen und lassen sich in 5 Teilaufgaben unterteilen.
\begin{itemize}
    \item \gls{bcd}-zu-7-Segment Decoder
    \item Ripple-Carry-Addierer
    \item Borrow-Bit-Subtrahierer
    \item Ergebnisumschalter
    \item Binär-zu-\gls{bcd} Umwandler
\end{itemize}
Für das Abarbeiten dieser Aufgaben ist das Labor da, falls die Zeit nicht reicht muss noch nachgearbeitet werden. Anschliessend soll ein Projektbericht erstellt werden.

\subsubsection{Vorgehen}
Das allgemeine Vorgehen ist für alle Aufgabenstellungen gleich. Als erstes wird für jede Baugruppe eine Wahrheitstabelle erstellt. Mit dieser Wahrheitstabelle
lässt sich das \gls{kv}-Diagramm ableiten. Mit dem erstellten \gls{kv}-Diagramm lässt sich die Normalform ablesen. Falls in der Aufgabenstellung gefordert, müssen
die Normalformen erweitert werden, so dass sie auf den beschränkten Gattervorrat passen. Die daraus resultierenden Logikfunktionen werden anschliessend in Logisim-evolution (\gls{le})
implementiert. Für dieses Projekt gab es bereits eine Vorlage die verwendet werden musste. Mithilfe der Simulation in \gls{le} kann die jeweilige Funktion direkt 
auf ihre richtige Funktionsweise überprüft werden. Zum Abschluss erfolgt die Verifikation mit dem \gls{fpga}-Board der BFH, dem Leguanboard (\gls{lb}). 
Dank der \gls{le}-Vorlage werden die erstellten Funktionen bereits automatisch in die jeweilige Testumgebung integriert. Es bedarf nur noch dem Download der \gls{le}-Datei und dem
festlegen der I/O-Elemente auf dem \gls{lb}. Anschliessend kann die Logikfunktion der jeweiligen Baugruppen mit physischen Elementen getestet werden.
Für alle Baugruppen wird ein Testprotokoll erstellt und ausgefüllt. Mehr dazu in Kapitel~\ref{sec:results}.

Die oben genannten Arbeitsschritte werden im Nachhinein dokumentiert. Die Ergebnisse werden in Kapitel~\ref{sec:results} dokumentiert.

\subsection{Projektorganisation}
Das Projekt wurde ressourcentechnisch wie folgt aufgeteilt:

\paragraph{Vorbereitungsaufgaben}
\begin{itemize}
    \item Funktionen für den \gls{bcd}-zu-7-Segment-Decoder ermitteln --> Simon Eisele
    \item Konzept für den Algorithmus zur Umwandlung einer 8 Bit-Binärzahl in 3 \gls{bcd}-Ziffern --> Janis Aebischer
    \item \gls{le} kennelernen und über die Möglichkeiten dieser Software informieren. --> Janis Aebischer, Simon Eisele
\end{itemize}

\paragraph{Aufgaben}
\begin{itemize}
    \item \gls{bcd}-zu-7-Segment Decoder --> Simon Eisele
    \item Ripple-Carry-Addierer --> Simon Eisele
    \item Borrow-Bit-Subtrahierer --> Simon Eisele
    \item Ergebnisumschalter --> Janis Aebischer
    \item Binär-zu-\gls{bcd} Umwandler --> Janis Aebischer
\end{itemize}

\paragraph{Sonstiges}
\begin{itemize}
    \item Dokumentation --> Janis Aebischer, Simon Eisele
    \item Erstellen \gls{kv}-Diagramme --> Janis Aebischer
    \item Aufsetzen von \gls{latex} --> Simon Eisele
    \item Erstellen Testprotokolle --> Janis Aebischer, Simon Eisele
\end{itemize}

%----------------Versuchs- und Projektaufbau--------------------------------------------------------------------------------------------------------------------------------------------------------
\subsection{Verwendete Materialien, Werkzeuge und Software}
\paragraph{Entwerfen und Testen der Schaltungen}
Für den Entwurf der verschiedenen Schaltungen wurde die Software Logisim-evolution (\gls{le})~\cite{burchLogisimevolution2024} verwendet. Mit dieser lassen sich die Schaltungen aus Grundgattern aufbauen, sowie auch simulieren.
Für die praktischen Tests wurde das Leguan-Board (\gls{lb})~\cite{HomeLeguanDevelopers} der \gls{bfh} verwendet.

\paragraph{Projektbericht}
Der Projektbericht wurde mithilfe von \gls{latex} verfasst. Um bereits eine passende Grundlage zu haben, wurde eine Vorlage der \gls{bfh} verwendet und entsprechend unserer Bedürfnisse
angepasst.~\cite{IntroductionLaTeXManual}

%----------------Test- und Verifikationsverfahren---------------------------------------------------------------------------------------------------------------------------------------------------
\subsection{Test- und Verifikationsverfahren}
Jede entwickelte Schaltung wird in der Simulation und auf der Hardware überprüft. Die Simulation erfolgt in \gls{le}. Auf dem \gls{lb} erfolgt die Verifikation der Funktion manuell
durch betätigen der DIP-Schalter und visuell über die 7-Segement-Anzeigen. Als Anzeige für \textit{overflow} oder \textit{underflow} werden die verbauten LED's auf dem \gls{lb} verwendet.
Für den Test grösserer Schaltungen eignen sich vorwiegend Werte, mit welchen die Sonderfälle abgedeckt werden können. Diese sind Überlauf, Unterlauf und Grenzwerte der \gls{bcd}-Darstellung. Die Ergebnisse sind in Kapitel \ref{sec:results} ersichtlich.
Alle Tests wurden zuerst in Excel festgehalten und anschliessend in \gls{latex} übernommen.

%----------------Spezifische Methoden pro Baugruppe-------------------------------------------------------------------------------------------------------------------------------------------------
\subsection{Spezifische Methoden pro Baugruppe}
Spezifische Methodik für bestimmte Baugruppen, falls etwas nicht dem Standardschema folgt.

\subsubsection{BCD-zu-7-Segment Decoder}

\paragraph{7-Segment-Anzeige}
Die 7-Segment-Anzeige besteht aus sieben einzeln ansteuerbaren Leuchtsegmenten, welche in Form einer Acht angeordnet sind. Durch unterschiedliche Ansteuerung dieser acht Segmente
lassen sich die Ziffern 0 bis 9 darstellen. Die Segmente werden mit den Buchstaben A bis G bezeichnet. Zusätzlich gibt es noch ein 9. Segment, mit welchem sich ein Punkt nach der Ziffer
darstellen lässt. Dieses wird aber im folgenden ignoriert, da es für die Aufgaben nicht benötigt wird und irrelevant ist.
\begin{figure}[H]
    \centering
    \includegraphics[width=0.25\textwidth]{bcd_to_7segment/7-segement-display.png}
    \caption{Anordnung der Segmente einer 7-Segment-Anzeige~\cite[Abb.~1]{tamselKombinatorischeLogikProjekt2025}}
    \label{fig:7-segemnt-display}
\end{figure}

\paragraph{Wahrheitstabelle}
\noindent Aus den Abbildungen~\ref{fig:7-segemnt-display} und~\ref{fig:numbers_on_a_7-segment-display} ist ersichtlich, welche Segmente A bis G zur Darstellung der einzelnen Ziffern aktiviert werden müssen.
Zur Ansteuerung der 7-Segment-Anzeige werden vier Bits (ein Nibble) verwendet, womit die Dezimalzahlen 0 bis 15 codierbar sind. Da die Anzeige nur die Ziffern 0 bis 9 darstellen kann, wurden die 
verbleibenden Werte gemäß Aufgabenstellung~\cite{tamselKombinatorischeLogikProjekt2025} als „don’t care“ definiert.
\begin{figure}[H]
    \centering
    \includegraphics[width=0.9\textwidth]{bcd_to_7segment/numbers_on_a_7-segement-display.png}
    \caption{Darstellung der Ziffern 0 bis 9 auf der 7-Segment-Anzeige~\cite[Abb.~2]{tamselKombinatorischeLogikProjekt2025}}
    \label{fig:numbers_on_a_7-segment-display}
\end{figure}

\paragraph{KV-Diagramm}
In der Aufgabenstellung sind spezifische Normalformen verlangt. Diese sind entweder konjunktiv oder disjunktiv, welche genau ist aus der Aufgabenstellung zu entnehmen.~\cite[pp.~4-5]{tamselKombinatorischeLogikProjekt2025}

\paragraph{Beschränkter Gattervorrat}
Die aus den \gls{kv}-Diagrammen abgeleiteten Normalformen (siehe Tabelle~\ref{tab:kv_bcd_to_7segment}) wurden gemäß den Vorgaben der Aufgabenstellung~\cite[pp.~4–5]{tamselKombinatorischeLogikProjekt2025} angepasst.
Für die Realisierung der Schaltungen stand nur ein begrenzter Satz an Grundgattern zur Verfügung. So durfte beispielsweise das Segment A ausschließlich mit den Gattern \textsc{Xor} und \textsc{Or} umgesetzt werden~\cite[p.~4]{tamselKombinatorischeLogikProjekt2025}.
Die Umformungen der Normalformen erfolgten mithilfe der Booleschen Algebra und der De-Morgan-Theoreme.

\subsubsection{Ripple-Carry-Addierer}
Um den Ripple-Carry-Addierer zu realisieren, muss zuerst der Volladdierer gebaut werden. Anschliessend wird die erstellte Baugruppe für das Realisieren der Ripple-Carry-Addierer Baugruppe verwendet.
Der Volladdierer wird nicht einzeln in einem Testprotokoll getestet, da er im kombinierten Test der übergeordneten Baugruppe automatisch getestet wird.

\paragraph{Volladdierer}
Der Ripple-Carry-Addierer besteht aus mehreren, hintereinandergeschalteten Volladdierern. Ein Volladdierer verrechnet dabei jeweils drei Bits miteinander. Je ein Bit
der jeweiligen Summanden, sowie ein Übertragsbit des vorherigen Volladdierers. Das Ganze funktioniert also gleich, wie auch die schriftliche Addition. Der Volladdierer liefert dabei die Summe,
welche als Ergebnis an die jeweilige Stelle kommt, sowie ein Übertragsbit, welches an den Volladdierer der nächsten Stelle übergeben wird.

\paragraph{Ripple-Carry-Addierer}
Der Ripple-Carry-Addierer wird für die Addition zweier Binärzahlen verwendet. Er besteht aus mehreren Volladdierern, und kann
für eine unbegrenzte Anzahl Bits verwendet werden. Der Name \glqq Ripple\grqq~kommt daher, da bei der Berechnung jeweils ein Übertragsbit von einem Volladdierer zum nächsten durchgereicht wird.
Dadurch steigt aber auch die Latenz mit zunehmender Anzahl Bits, da jeder Volladdierer auf den Übertrag des letzten warten muss, bevor die Berechnung durchgeführt werden kann.~\cite{manoDigitalDesignIntroduction}

\paragraph{8-Bit-Addierer}
Aus dem entwickelten Volladdierer lassen sich nun beliebig grosse Addierer realisieren. Gemäss Aufgabenstellung~\cite{tamselKombinatorischeLogikProjekt2025} soll
ein 8-Bit-Addierer gebaut werden. Dies funktioniert durch Aneinanderreihung der Volladdierer und jeweiliger Übergabe des Übertrags auf den darauffolgenden
Volladdierer. Der Übertragsausgang des letzten Volladdierers wurde an den Ausgang \glqq Overflow\grqq~angeschlossen. Falls also die Summe der Addition zu gross ist, um
noch korrekt dargestellt werden zu können, wird dies rückgemeldet.

\subsubsection{Borrow-Bit-Subtrahierer}
Um den Borrow-Bit-Subtrahierer zu realisieren, muss zuerst der Vollsubtrahierer gebaut werden. Anschliessend wird die erstellte Baugruppe für das Realisieren der Borrow-Bit-Subtrahierer Baugruppe verwendet.
Der Vollsubtrahierer wird nicht einzeln in einem Testprotokoll getestet, da er im kombinierten Test der übergeordneten Baugruppe automatisch getestet wird.

\paragraph{Vollsubtrahierer}
Der Borrow-Bit-Subtrahierer besteht aus mehreren hintereinandergeschalteten Vollsubtrahierern. Jeder Vollsubtrahierer verarbeitet ein Bit des Minuenden, ein Bit des Subtrahenden,
sowie das Borrow-Bit (Übertragsbit) des vorherigen Vollsubtrahierers. Der Vollasubtrahierer liefert dabei das Differenzbit, welches als Ergebnis an die jeweilige Stelle kommt,
sowie ein Borrow-Bit, welches an den Vollsubtrahierer der nächsten Stelle übergeben wird, falls von dieser ein Bit geborgt werden muss. Die Funktionsweise entspricht damit der schriftlichen
Subtraktion, bei der, bei Bedarf, von der nächsthöheren Stelle \glqq ausgeliehen\grqq~wird.

\paragraph{8-Bit-Subtrahierer}
Aus dem entwickelten Vollsubtrahierer lassen sich nun beliebig grosse Subtrahierer realisieren. Gemäss Aufgabenstellung~\cite{tamselKombinatorischeLogikProjekt2025} soll
ein 8-Bit-Subtrahierer gebaut werden. Dies funktioniert durch Aneinanderreihung der Vollsubtrahierer und jeweiliger Übergabe des Übertrags auf den darauffolgenden
Vollsubtrahierer. Der Übertragsausgang des letzten Vollsubtrahierers wurde an den Ausgang \glqq Underflow\grqq~angeschlossen. Falls also die Differenz der Subtraktion kleiner als 0 ist,
wird dies rückgemeldet, so dass klar ist, dass das angezeigt Ergebnis nicht die korrekte Lösung der Rechnung ist.

\subsubsection{Ergebnisumschalter}
Mit dem gleichen Prinzip wie auch bei den vorherigen Schaltungen, ist der 8-Bit-Multiplexer aus mehreren 1-Bit-Multiplexern aufgebaut.
Daher muss zuerst der 1-Bit-Multiplexer realisiert werden. Dieser wird nicht einzeln in einem Testprotokoll getestet, da er im kombinierten Test der übergeordneten Baugruppe automatisch getestet wird.

\subsubsection{Binär-zu-BCD Umwandler}
Eine 7-Segment-Anzeige kann maximal 10 Ziffern anzeigen. Bei einer 8-Bit-Binärzahl kann somit nur ein Bruchteil ihres Wertes wiedergegeben werden. Damit eine 8-Bit-Binärzahl
in ihrer gesamtheitlichen Grösse wiedergegeben werden kann, sind 3 Ziffern nötig. Aus der 8-Bit-Binärzahl muss also für jede Ziffer der entsprechende \gls{bcd}-Wert ermittelt werden.
Wie genau dies gemacht wird kann aus der Vorbereitungsaufgabe entnommen werden. (Siehe Anhang~\ref{app:pdf}) Durch geeignetes Verknüpfen der erstellten Add3-Blöcke lässt sich die gewünschte Funktion erstellen.

\paragraph{Add3}
Sobald eine Zahl >=5 ist, wird der Wert 3 (0011 in binär) zu dieser Zahl dazugerechnet.

%----------------  Results   -----------------------------------------------
\clearpage
\section{Resultate}
\label{sec:results}
%----------------  BCD-zu-7-Segment-Decoder   ------------------------------------------------------------------------------------------------------------------------------------------------------
\subsection{BCD-zu-7-Segment-Decoder}
\label{sub:results_bcd-to-7-segment-decoder}
\subsubsection{Wahrheitstabelle}
\begin{table}[H]
\caption{Wahrheitstabelle Siebensegmentanzeige}\label{tab:truthtable_bcd_to_7segment}
\centering
\colorlet{BFH-table}{BFH-MediumBlue!10}
\colorlet{BFH-tablehead}{BFH-MediumBlue!50}
\setupBfhTabular
\begin{bfhTabular}{cccccccccccc}
Input & a3 & a2 & a1 & a0 & seg\_A & seg\_B & seg\_C & seg\_D & seg\_E & seg\_F & seg\_G \\\hline
\num{0} & \num{0} & \num{0} & \num{0} & \num{0} & \num{1} & \num{1} & \num{1} & \num{1} & \num{1} & \num{1} & \num{0} \\\hline
\num{1} & \num{0} & \num{0} & \num{0} & \num{1} & \num{0} & \num{1} & \num{1} & \num{0} & \num{0} & \num{0} & \num{0} \\\hline
\num{2} & \num{0} & \num{0} & \num{1} & \num{0} & \num{1} & \num{1} & \num{0} & \num{1} & \num{1} & \num{0} & \num{1} \\\hline
\num{3} & \num{0} & \num{0} & \num{1} & \num{1} & \num{1} & \num{1} & \num{1} & \num{1} & \num{0} & \num{0} & \num{1} \\\hline
\num{4} & \num{0} & \num{1} & \num{0} & \num{0} & \num{0} & \num{1} & \num{1} & \num{0} & \num{0} & \num{1} & \num{1} \\\hline
\num{5} & \num{0} & \num{1} & \num{0} & \num{1} & \num{1} & \num{0} & \num{1} & \num{1} & \num{0} & \num{1} & \num{1} \\\hline
\num{6} & \num{0} & \num{1} & \num{1} & \num{0} & \num{1} & \num{0} & \num{1} & \num{1} & \num{1} & \num{1} & \num{1} \\\hline
\num{7} & \num{0} & \num{1} & \num{1} & \num{1} & \num{1} & \num{1} & \num{1} & \num{0} & \num{0} & \num{0} & \num{0} \\\hline
\num{8} & \num{1} & \num{0} & \num{0} & \num{0} & \num{1} & \num{1} & \num{1} & \num{1} & \num{1} & \num{1} & \num{1} \\\hline
\num{9} & \num{1} & \num{0} & \num{0} & \num{1} & \num{1} & \num{1} & \num{1} & \num{1} & \num{0} & \num{1} & \num{1} \\\hline
\num{10} & \num{1} & \num{0} & \num{1} & \num{0} & \num{0} & - & - & - & - & - & - \\\hline
\num{11} & \num{1} & \num{0} & \num{1} & \num{1} & \num{0} & - & - & - & - & - & - \\\hline
\num{12} & \num{1} & \num{1} & \num{0} & \num{0} & \num{0} & - & - & - & - & - & - \\\hline
\num{13} & \num{1} & \num{1} & \num{0} & \num{1} & \num{0} & - & - & - & - & - & - \\\hline
\num{14} & \num{1} & \num{1} & \num{1} & \num{0} & \num{0} & - & - & - & - & - & - \\\hline
\num{15} & \num{1} & \num{1} & \num{1} & \num{1} & \num{0} & - & - & - & - & - & - \\\hline
\end{bfhTabular}
\end{table}

\subsubsection{\gls{kv}-Diagramme}
\begin{longtable}{lll}
\caption{\gls{kv}-Diagramme und Normalfunktion der verschiedenen Segmente}\label{tab:kv_bcd_to_7segment} \\
\hline
\rowcolor{BFH-MediumRed!50} Segment & KV-Diagramm & Funktion \rule{0pt}{1.3em} \\
\hline
\endfirsthead
\multicolumn{3}{l}{$...$~Fortsetzung von Tabelle \ref{tab:kv_bcd_to_7segment}} \\
\hline
\rowcolor{BFH-MediumRed!50} Segment & KV-Diagramm & Funktion \rule{0pt}{1.3em} \\
\hline
\endhead
\hline \multicolumn{3}{r}{{Fortsetzung auf nächster Seite$...$}} \\
\hline
\endfoot
\hline \hline
\endlastfoot
\rowcolor{BFH-MediumRed!10}
A & \adjustbox{valign=t}{\includegraphics[width=0.3\textwidth]{bcd_to_7segment/kv_segement_A.png}} & $Y=\left(a_0+a_1+\overline{a_2}\right)\cdot(\overline{a_0}+a_1+a_2+a_3)$ \\\hline
\rowcolor{BFH-MediumRed!10}
B & \adjustbox{valign=t}{\includegraphics[width=0.3\textwidth]{bcd_to_7segment/kv_segement_B.png}} & $Y=(\overline{a_0}+a_1+\overline{a_2})\cdot(a_0+\overline{a_1}+\overline{a_2})$ \\\hline
\rowcolor{BFH-MediumRed!10}
C & \adjustbox{valign=t}{\includegraphics[width=0.3\textwidth]{bcd_to_7segment/kv_segement_C.png}} & $Y=a_0+\overline{a_1}+a_2$ \\\hline
\rowcolor{BFH-MediumRed!10}
D & \adjustbox{valign=t}{\includegraphics[width=0.3\textwidth]{bcd_to_7segment/kv_segement_D.png}} & $Y=(a_0+a_1+\overline{a_2})\cdot(\overline{a_0}+\overline{a_1}+\overline{a_2})\cdot(\overline{a_0}+a_1+a_2+a_3)$ \\\hline
\rowcolor{BFH-MediumRed!10}
E & \adjustbox{valign=t}{\includegraphics[width=0.3\textwidth]{bcd_to_7segment/kv_segement_E.png}} & $Y=\overline{a_0}\cdot\overline{a_2}+\overline{a_0}\cdot a_1$ \\\hline
\rowcolor{BFH-MediumRed!10}
F & \adjustbox{valign=t}{\includegraphics[width=0.3\textwidth]{bcd_to_7segment/kv_segement_F.png}} & $Y=a_3+\overline{a_0}\cdot\overline{a_1}+\overline{a_0}\cdot a_2+\overline{a_1}\cdot a_2$ \\\hline
\rowcolor{BFH-MediumRed!10}
G & \adjustbox{valign=t}{\includegraphics[width=0.3\textwidth]{bcd_to_7segment/kv_segement_G.png}} & $Y=(a_1+a_2+a_3)\cdot(\overline{a_0}+\overline{a_1}+\overline{a_2})$ \\\hline
\end{longtable}

\subsubsection{Beschränkter Gattervorrat}
\begin{table}[H]
\caption{Funktionen mit beschränktem Gattervorrat}\label{tab:limited_bcd_to_7segment}
\centering
\colorlet{BFH-table}{BFH-MediumRed!10}
\colorlet{BFH-tablehead}{BFH-MediumRed!50}
\setupBfhTabular
\resizebox{\textwidth}{!}{
\begin{bfhTabular}{lll}
Segment & Gattervorrat & Funktion \\\hline
A & \textsc{Xor} und \textsc{Or} & $Y=a_1+a_3+((a_0\oplus a_2)\oplus 1)$ \\\hline
B & \textsc{Nor} & $Y=\overline{\overline{\overline{a_0+a_0}+a_1+\overline{a_2+a_2}}+\overline{a_0+\overline{a_1+a_1}+\overline{a_2+a_2}}}$ \\\hline
C & \textsc{And}, \textsc{Or} und \textsc{Not} & $Y=a_0+\overline{a_1}+a_2$ \\\hline
D & \textsc{And}, \textsc{Or} und \textsc{Not} & $Y=(a_0+a_1+\overline{a_2})\cdot(\overline{a_0}+\overline{a_1}+\overline{a_2})\cdot(\overline{a_0}+a_1+a_2+a_3)$ \\\hline
E & \textsc{And} und \textsc{Xor} & $Y=(a_0\oplus 1)\cdot(((a_1\oplus 1)\cdot a_2)\oplus 1)$ \\\hline
F & \textsc{Nand} & $Y=\overline{\overline{a_3\cdot a_3}\cdot\overline{\overline{a_0\cdot a_0}\cdot\overline{a_1\cdot a_1}}\cdot\overline{\overline{a_0\cdot a_0}\cdot a_2}\cdot\overline{\overline{a_1\cdot a_1}\cdot a_2}}$ \\\hline
G & \textsc{And2}, \textsc{Or2} und \textsc{Not} & $Y=((a_1+a_2)+a_3)\cdot((\overline{a_0}+\overline{a_1})+\overline{a_2})$ \\\hline
\end{bfhTabular}
}
\end{table}

\subsubsection{Implementierung in Logisim-Evolution}
\begin{longtable}{ll}
\caption{Implementierung der Funktionen in \gls{le}}\label{tab:logisim_bcd_to_7segment} \\
\hline
\rowcolor{BFH-MediumRed!50} Schaltungsbezeichnung & Schaltung \rule{0pt}{1.3em} \\
\hline
\endfirsthead
\multicolumn{2}{l}{$...$~Fortsetzung von Tabelle \ref{tab:logisim_bcd_to_7segment}} \\
\hline
\rowcolor{BFH-MediumRed!50} Schaltungsbezeichnung & Schaltung \rule{0pt}{1.3em} \\
\hline
\endhead
\hline \multicolumn{2}{r}{{Fortsetzung auf nächster Seite$...$}} \\
\hline
\endfoot
\hline \hline
\endlastfoot
\rowcolor{BFH-MediumRed!10}
segmentA & \adjustbox{valign=t}{\includegraphics[width=0.7\textwidth]{bcd_to_7segment/segment_A.png}} \\\hline
\rowcolor{BFH-MediumRed!10}
segmentB & \adjustbox{valign=t}{\includegraphics[width=0.7\textwidth]{bcd_to_7segment/segment_B.png}} \\\hline
\rowcolor{BFH-MediumRed!10}
segmentC & \adjustbox{valign=t}{\includegraphics[width=0.7\textwidth]{bcd_to_7segment/segment_C.png}} \\\hline
\rowcolor{BFH-MediumRed!10}
segmentD & \adjustbox{valign=t}{\includegraphics[width=0.7\textwidth]{bcd_to_7segment/segment_D.png}} \\\hline
\rowcolor{BFH-MediumRed!10}
segmentE & \adjustbox{valign=t}{\includegraphics[width=0.7\textwidth]{bcd_to_7segment/segment_E.png}} \\\hline
\rowcolor{BFH-MediumRed!10}
segmentF & \adjustbox{valign=t}{\includegraphics[width=0.7\textwidth]{bcd_to_7segment/segment_F.png}} \\\hline
\rowcolor{BFH-MediumRed!10}
segmentG & \adjustbox{valign=t}{\includegraphics[width=0.7\textwidth]{bcd_to_7segment/segment_G.png}} \\\hline
\end{longtable}

\subsubsection{Test der Schaltungen}
\begin{table}[H]
\caption{Testprotokoll Siebensegmentanzeige}\label{tab:test_protocol_bcd_to_7segment}
\centering
\colorlet{BFH-table}{BFH-MediumGreen!10}
\colorlet{BFH-tablehead}{BFH-MediumGreen!50}
\setupBfhTabular
\resizebox{\textwidth}{!}{
\begin{bfhTabular}{cc|ccc|ccc|ccc|ccc|ccc|ccc|ccc}
\multicolumn{2}{c}{Input} & \multicolumn{3}{c}{Segment A} & \multicolumn{3}{c}{Segment B} & \multicolumn{3}{c}{Segment C} & \multicolumn{3}{c}{Segment D} & \multicolumn{3}{c}{Segment E} & \multicolumn{3}{c}{Segment F} & \multicolumn{3}{c}{Segment G} \\\hline
Digitalwert & Binärwert & E & \gls{le} & \gls{lb} & E & \gls{le} & \gls{lb} & E & \gls{le} & \gls{lb} & E & \gls{le} & \gls{lb} & E & \gls{le} & \gls{lb} & E & \gls{le} & \gls{lb} & E & \gls{le} & \gls{lb} \\\hline
\num{0} & \num[minimum-integer-digits=4]{0000} & \num{1} & \num{1} & \num{1} & \num{1} & \num{1} & \num{1} & \num{1} & \num{1} & \num{1} & \num{1} & \num{1} & \num{1} & \num{1} & \num{1} & \num{1} & \num{1} & \num{1} & \num{1} & \num{0} & \num{0} & \num{0} \\\hline
\num{1} & \num[minimum-integer-digits=4]{0001} & \num{0} & \num{0} & \num{0} & \num{1} & \num{1} & \num{1} & \num{1} & \num{1} & \num{1} & \num{0} & \num{0} & \num{0} & \num{0} & \num{0} & \num{0} & \num{0} & \num{0} & \num{0} & \num{0} & \num{0} & \num{0} \\\hline
\num{2} & \num[minimum-integer-digits=4]{0010} & \num{1} & \num{1} & \num{1} & \num{1} & \num{1} & \num{1} & \num{0} & \num{0} & \num{0} & \num{1} & \num{1} & \num{1} & \num{1} & \num{1} & \num{1} & \num{0} & \num{0} & \num{0} & \num{1} & \num{1} & \num{1} \\\hline
\num{3} & \num[minimum-integer-digits=4]{0011} & \num{1} & \num{1} & \num{1} & \num{1} & \num{1} & \num{1} & \num{1} & \num{1} & \num{1} & \num{1} & \num{1} & \num{1} & \num{0} & \num{0} & \num{0} & \num{0} & \num{0} & \num{0} & \num{1} & \num{1} & \num{1} \\\hline
\num{4} & \num[minimum-integer-digits=4]{0100} & \num{0} & \num{0} & \num{0} & \num{1} & \num{1} & \num{1} & \num{1} & \num{1} & \num{1} & \num{0} & \num{0} & \num{0} & \num{0} & \num{0} & \num{0} & \num{1} & \num{1} & \num{1} & \num{1} & \num{1} & \num{1} \\\hline
\num{5} & \num[minimum-integer-digits=4]{0101} & \num{1} & \num{1} & \num{1} & \num{0} & \num{0} & \num{0} & \num{1} & \num{1} & \num{1} & \num{1} & \num{1} & \num{1} & \num{0} & \num{0} & \num{0} & \num{1} & \num{1} & \num{1} & \num{1} & \num{1} & \num{1} \\\hline
\num{6} & \num[minimum-integer-digits=4]{0110} & \num{1} & \num{1} & \num{1} & \num{0} & \num{0} & \num{0} & \num{1} & \num{1} & \num{1} & \num{1} & \num{1} & \num{1} & \num{1} & \num{1} & \num{1} & \num{1} & \num{1} & \num{1} & \num{1} & \num{1} & \num{1} \\\hline
\num{7} & \num[minimum-integer-digits=4]{0111} & \num{1} & \num{1} & \num{1} & \num{1} & \num{1} & \num{1} & \num{1} & \num{1} & \num{1} & \num{0} & \num{0} & \num{0} & \num{0} & \num{0} & \num{0} & \num{0} & \num{0} & \num{0} & \num{0} & \num{0} & \num{0} \\\hline
\num{8} & \num[minimum-integer-digits=4]{1000} & \num{1} & \num{1} & \num{1} & \num{1} & \num{1} & \num{1} & \num{1} & \num{1} & \num{1} & \num{1} & \num{1} & \num{1} & \num{1} & \num{1} & \num{1} & \num{1} & \num{1} & \num{1} & \num{1} & \num{1} & \num{1} \\\hline
\num{9} & \num[minimum-integer-digits=4]{1001} & \num{1} & \num{1} & \num{1} & \num{1} & \num{1} & \num{1} & \num{1} & \num{1} & \num{1} & \num{1} & \num{1} & \num{1} & \num{0} & \num{0} & \num{0} & \num{1} & \num{1} & \num{1} & \num{1} & \num{1} & \num{1} \\\hline
\num{10} & \num[minimum-integer-digits=4]{1010} & - & \num{1} & \num{1} & - & \num{1} & \num{1} & - & \num{0} & \num{0} & - & \num{1} & \num{1} & - & \num{1} & \num{1} & - & \num{1} & \num{1} & - & \num{1} & \num{1} \\\hline
\num{11} & \num[minimum-integer-digits=4]{1011} & - & \num{1} & \num{1} & - & \num{1} & \num{1} & - & \num{1} & \num{1} & - & \num{1} & \num{1} & - & \num{0} & \num{0} & - & \num{1} & \num{1} & - & \num{1} & \num{1} \\\hline
\num{12} & \num[minimum-integer-digits=4]{1100} & - & \num{1} & \num{1} & - & \num{1} & \num{1} & - & \num{1} & \num{1} & - & \num{0} & \num{0} & - & \num{0} & \num{0} & - & \num{1} & \num{1} & - & \num{1} & \num{1} \\\hline
\num{13} & \num[minimum-integer-digits=4]{1101} & - & \num{1} & \num{1} & - & \num{0} & \num{0} & - & \num{1} & \num{1} & - & \num{1} & \num{1} & - & \num{0} & \num{0} & - & \num{1} & \num{1} & - & \num{1} & \num{1} \\\hline
\num{14} & \num[minimum-integer-digits=4]{1110} & - & \num{1} & \num{1} & - & \num{0} & \num{0} & - & \num{1} & \num{1} & - & \num{1} & \num{1} & - & \num{1} & \num{1} & - & \num{1} & \num{1} & - & \num{1} & \num{1} \\\hline
\num{15} & \num[minimum-integer-digits=4]{1111} & - & \num{1} & \num{1} & - & \num{1} & \num{1} & - & \num{1} & \num{1} & - & \num{0} & \num{0} & - & \num{0} & \num{0} & - & \num{1} & \num{1} & - & \num{0} & \num{0} \\\hline
\end{bfhTabular}
}
\footnotesize
\textit{Legende:} E = Erwartet, LE = Simuliert in Logisim-evolution, LB = Getestet auf Leguan-Board
\end{table}

%----------------  Ripple-Carry-Addierer   ---------------------------------------------------------------------------------------------------------------------------------------------------------
\subsection{Ripple-Carry-Addierer}
\label{sub:results_ripple-carry-adder}
\subsubsection{Wahrheitstabelle}
\begin{table}[H]
\caption{Wahrheitstabelle Volladdierer}\label{tab:truthtable_fulladder}
\centering
\colorlet{BFH-table}{BFH-MediumBlue!10}
\colorlet{BFH-tablehead}{BFH-MediumBlue!50}
\setupBfhTabular
\begin{bfhTabular}{ccccc}
Xin & Yin & Cin & Sum & Cout \\\hline
\num{0} & \num{0} & \num{0} & \num{0} & \num{0} \\\hline
\num{0} & \num{0} & \num{1} & \num{1} & \num{0} \\\hline
\num{0} & \num{1} & \num{0} & \num{1} & \num{0} \\\hline
\num{0} & \num{1} & \num{1} & \num{0} & \num{1} \\\hline
\num{1} & \num{0} & \num{0} & \num{1} & \num{0} \\\hline
\num{1} & \num{0} & \num{1} & \num{0} & \num{1} \\\hline
\num{1} & \num{1} & \num{0} & \num{0} & \num{1} \\\hline
\num{1} & \num{0} & \num{1} & \num{1} & \num{1} \\\hline
\end{bfhTabular}
\end{table}
\subsubsection{\gls{kv}-Diagramme}
\begin{table}[H]
\caption{KV-Diagramme und konjunktive Normalfunktion der Ausgänge des Volladdierers}
\label{tab:kv_fulladder}
\centering
\colorlet{BFH-table}{BFH-MediumRed!10}
\colorlet{BFH-tablehead}{BFH-MediumRed!50}
\setupBfhTabular
\begin{bfhTabular}{lll}
Output & KV-Diagramm & konjunktive Normalform \\\hline
Sum & \adjustbox{valign=t}{\includegraphics[width=0.3\textwidth]{ripple-carry-adder/sum.png}} & $Sum=X_{in}\oplus Y_{in} \oplus C_{in}$ \\\hline
Cout & \adjustbox{valign=t}{\includegraphics[width=0.3\textwidth]{ripple-carry-adder/cout.png}} & $Cout=(X_{in}\cdot Y_{in})+(X_{in}\cdot C_{in})+(Y_{in}\cdot C_{in})$ \\\hline
\end{bfhTabular}
\end{table}
\subsubsection{Implementierung in Logisim-evolution}
\begin{figure}[H]
    \centering
    \includegraphics[width=1\textwidth]{ripple-carry-adder/fulladder.png}
    \caption{Implementierung des Volladdierers in \gls{le}}
    \label{fig:logisim_fulladder}
\end{figure}
\subsubsection{8-Bit-Addierer}
\begin{figure}[H]
    \centering
    \includegraphics[width=1\textwidth]{ripple-carry-adder/8-bit-adder.png}
    \caption{Aufbau des 8-Bit-Addierers in \gls{le}}
    \label{fig:logisim_8-bit-adder}
\end{figure}
\subsubsection{Test der Schaltungen}
\begin{figure}[H]
    \centering
    \includegraphics[width=1\textwidth]{ripple-carry-adder/adder_test_bench.png}
    \caption{Schaltung AdderTestBench in \gls{le}}
    \label{fig:logisim_adder_test_bench}
\end{figure}
\begin{table}[H]
\caption{Testprotokoll 8-Bit-Addierer}\label{tab:test_protocol_8-bit-adder}
\centering
\colorlet{BFH-table}{BFH-MediumGreen!10}
\colorlet{BFH-tablehead}{BFH-MediumGreen!50}
\setupBfhTabular
\resizebox{\textwidth}{!}{
\begin{bfhTabular}{cc|cc|c|c|c}
\multicolumn{2}{c}{Input Summand 1}                                                     & \multicolumn{2}{c}{Input Summand 2}                                                       & Erwarteter Wert       & Resultat Simulation   & Resultat Leguan-Board \\\hline
Digitalwert & Binärwert                                                                 & Digitalwert   & Binärwert                                                                 &                       &                       &                       \\\hline
\num{0}     & \num[minimum-integer-digits=4]{0000} \num[minimum-integer-digits=4]{0000} & \num{0}       & \num[minimum-integer-digits=4]{0000} \num[minimum-integer-digits=4]{0000} & \num{0}               & \num{0}               & \num{0}               \\\hline
\num{255}   & \num[minimum-integer-digits=4]{1111} \num[minimum-integer-digits=4]{1111} & \num{0}       & \num[minimum-integer-digits=4]{0000} \num[minimum-integer-digits=4]{0000} & \num{255}             & \num{255}             & \num{255}             \\\hline
\num{255}   & \num[minimum-integer-digits=4]{1111} \num[minimum-integer-digits=4]{1111} & \num{1}       & \num[minimum-integer-digits=4]{0000} \num[minimum-integer-digits=4]{0001} & overflow+\num{0}      & overflow+\num{0}      & overflow+\num{0}      \\\hline
\num{255}   & \num[minimum-integer-digits=4]{1111} \num[minimum-integer-digits=4]{1111} & \num{255}     & \num[minimum-integer-digits=4]{1111} \num[minimum-integer-digits=4]{1111} & overflow+\num{254}    & overflow+\num{254}    & overflow+\num{254}    \\\hline
\num{240}   & \num[minimum-integer-digits=4]{1111} \num[minimum-integer-digits=4]{0000} & \num{16}      & \num[minimum-integer-digits=4]{0001} \num[minimum-integer-digits=4]{0000} & overflow+\num{0}      & overflow+\num{0}      & overflow+\num{0}      \\\hline
\num{15}    & \num[minimum-integer-digits=4]{0000} \num[minimum-integer-digits=4]{1111} & \num{1}       & \num[minimum-integer-digits=4]{0000} \num[minimum-integer-digits=4]{0001} & \num{16}              & \num{16}              & \num{16}              \\\hline
\num{170}   & \num[minimum-integer-digits=4]{1010} \num[minimum-integer-digits=4]{1010} & \num{85}      & \num[minimum-integer-digits=4]{0101} \num[minimum-integer-digits=4]{0101} & \num{255}             & \num{255}             & \num{255}             \\\hline
\num{127}   & \num[minimum-integer-digits=4]{0111} \num[minimum-integer-digits=4]{1111} & \num{18}      & \num[minimum-integer-digits=4]{0000} \num[minimum-integer-digits=4]{0001} & \num{128}             & \num{128}             & \num{128}             \\\hline
\num{98}    & \num[minimum-integer-digits=4]{0110} \num[minimum-integer-digits=4]{0010} & \num{8}       & \num[minimum-integer-digits=4]{0000} \num[minimum-integer-digits=4]{1000} & \num{106}             & \num{106}             & \num{106}             \\\hline
\num{254}3   & \num[minimum-integer-digits=4]{1111} \num[minimum-integer-digits=4]{1110} & \num{1}       & \num[minimum-integer-digits=4]{0000} \num[minimum-integer-digits=4]{0001} & \num{255}             & \num{255}             & \num{255}             \\\hline
\end{bfhTabular}
}
\end{table}

%----------------  Borrow-Bit-Subtrahierer   -------------------------------------------------------------------------------------------------------------------------------------------------------
\subsection{Borrow-Bit-Subtrahierer}
\label{sub:results_borrow-bit-subtractor}
\subsubsection{Wahrheitstabelle}
\begin{table}[H]
\caption{Wahrheitstabelle Vollsubtrahierer}
\label{tab:truthtable_fullsubtractor}
\centering
\colorlet{BFH-table}{BFH-MediumBlue!10}
\colorlet{BFH-tablehead}{BFH-MediumBlue!50}
\setupBfhTabular
\begin{bfhTabular}{ccccc}
Xin & Yin & Bin & Sum & Bout \\\hline
\num{0} & \num{0} & \num{0} & \num{0} & \num{0} \\\hline
\num{0} & \num{0} & \num{1} & \num{1} & \num{1} \\\hline
\num{0} & \num{1} & \num{0} & \num{1} & \num{1} \\\hline
\num{0} & \num{1} & \num{1} & \num{0} & \num{1} \\\hline
\num{1} & \num{0} & \num{0} & \num{1} & \num{0} \\\hline
\num{1} & \num{0} & \num{1} & \num{0} & \num{0} \\\hline
\num{1} & \num{1} & \num{0} & \num{0} & \num{0} \\\hline
\num{1} & \num{0} & \num{1} & \num{1} & \num{1} \\\hline
\end{bfhTabular}
\end{table}
\subsubsection{\gls{kv}-Diagramme}
\begin{table}[H]
\caption{KV-Diagramme und konjunktive Normalfunktion der Ausgänge des Vollsubtrahierers}\label{tab:kv_fullsubtractor}
\centering
\colorlet{BFH-table}{BFH-MediumRed!10}
\colorlet{BFH-tablehead}{BFH-MediumRed!50}
\setupBfhTabular
\begin{bfhTabular}{lll}
Output & KV-Diagramm & konjunktive Normalform \\\hline
Sum & \adjustbox{valign=t}{\includegraphics[width=0.3\textwidth]{borrow-bit-subtractor/sum.png}} & $Sum=X_{in}\oplus Y_{in} \oplus B_{in}$ \\\hline
Bout & \adjustbox{valign=t}{\includegraphics[width=0.3\textwidth]{borrow-bit-subtractor/bout.png}} & $Bout=(\overline{X_{in}}\cdot Y_{in})+(\overline{X_{in}}\cdot B_{in})+(Y_{in}\cdot B_{in})$ \\\hline
\end{bfhTabular}
\end{table}
\subsubsection{Implementierung in Logisim-evolution}
\begin{figure}[H]
    \centering
    \includegraphics[width=1\textwidth]{borrow-bit-subtractor/fullsubtractor.png}
    \caption{Implementierung des Vollsubtrahierers in \gls{le}}
    \label{fig:logisim_fullsubtractor}
\end{figure}
\subsubsection{8-Bit-Subtrahierer}
\begin{figure}[H]
    \centering
    \includegraphics[width=1\textwidth]{borrow-bit-subtractor/8-bit-subtractor.png}
    \caption{Aufbau des 8-Bit-Subtrahierers in \gls{le}}
    \label{fig:logisim_8-bit-subtractor}
\end{figure}
\subsubsection{Test der Schaltungen}
\begin{figure}[H]
    \centering
    \includegraphics[width=1\textwidth]{borrow-bit-subtractor/subtractor_test_bench.png}
    \caption{Schaltung SubtractorTestBench in \gls{le}}
    \label{fig:logisim_subtractor_test_bench}
\end{figure}
\begin{table}[H]
\caption{Testprotokoll 8-Bit-Subtrahierer}\label{tab:test_protocol_8-bit-subtractor}
\centering
\colorlet{BFH-table}{BFH-MediumGreen!10}
\colorlet{BFH-tablehead}{BFH-MediumGreen!50}
\setupBfhTabular
\resizebox{\textwidth}{!}{
\begin{bfhTabular}{cc|cc|c|c|c}
\multicolumn{2}{c}{Input Minuend}                                                       & \multicolumn{2}{c}{Input Subtrahend}                                                      & Erwarteter Wert       & Resultat Simulation   & Resultat Leguan-Board \\\hline
Digitalwert & Binärwert                                                                 & Digitalwert   & Binärwert                                                                 &                       &                       &                       \\\hline
\num{0}     & \num[minimum-integer-digits=4]{0000} \num[minimum-integer-digits=4]{0000} & \num{0}       & \num[minimum-integer-digits=4]{0000} \num[minimum-integer-digits=4]{0000} & \num{0}               & \num{0}               & \num{0}               \\\hline
\num{5}     & \num[minimum-integer-digits=4]{0000} \num[minimum-integer-digits=4]{0101} & \num{0}       & \num[minimum-integer-digits=4]{0000} \num[minimum-integer-digits=4]{0000} & \num{5}               & \num{5}               & \num{5}               \\\hline
\num{0}     & \num[minimum-integer-digits=4]{0000} \num[minimum-integer-digits=4]{0000} & \num{1}       & \num[minimum-integer-digits=4]{0000} \num[minimum-integer-digits=4]{0001} & underflow+\num{255}   & underflow+\num{255}   & underflow+\num{255}   \\\hline
\num{255}   & \num[minimum-integer-digits=4]{1111} \num[minimum-integer-digits=4]{1111} & \num{255}     & \num[minimum-integer-digits=4]{1111} \num[minimum-integer-digits=4]{1111} & \num{0}               & \num{0}               & \num{0}               \\\hline
\num{255}   & \num[minimum-integer-digits=4]{1111} \num[minimum-integer-digits=4]{0000} & \num{1}       & \num[minimum-integer-digits=4]{0000} \num[minimum-integer-digits=4]{0001} & \num{254}             & \num{254}             & \num{254}             \\\hline
\num{1}     & \num[minimum-integer-digits=4]{0000} \num[minimum-integer-digits=4]{0001} & \num{255}     & \num[minimum-integer-digits=4]{1111} \num[minimum-integer-digits=4]{1111} & underflow+\num{2}     & underflow+\num{2}     & underflow+\num{2}     \\\hline
\num{128}   & \num[minimum-integer-digits=4]{1000} \num[minimum-integer-digits=4]{0000} & \num{128}     & \num[minimum-integer-digits=4]{1000} \num[minimum-integer-digits=4]{0000} & \num{0}               & \num{0}               & \num{0}               \\\hline
\num{16}    & \num[minimum-integer-digits=4]{0001} \num[minimum-integer-digits=4]{0000} & \num{1}       & \num[minimum-integer-digits=4]{0000} \num[minimum-integer-digits=4]{0001} & \num{15}              & \num{15}              & \num{15}              \\\hline
\num{200}   & \num[minimum-integer-digits=4]{1100} \num[minimum-integer-digits=4]{1000} & \num{150}     & \num[minimum-integer-digits=4]{1001} \num[minimum-integer-digits=4]{0110} & \num{50}              & \num{50}              & \num{50}              \\\hline
\num{0}     & \num[minimum-integer-digits=4]{0000} \num[minimum-integer-digits=4]{0000} & \num{255}     & \num[minimum-integer-digits=4]{1111} \num[minimum-integer-digits=4]{1111} & underflow+\num{1}     & underflow+\num{1}     & underflow+\num{1}     \\\hline
\end{bfhTabular}
}
\end{table}

%----------------  Ergebnisumschalter   ------------------------------------------------------------------------------------------------------------------------------------------------------------
\subsection{Ergebnisumschalter}
\label{sub:results_multiplexer}
\subsubsection{Wahrheitstabelle}
\begin{table}[H]
\caption{Wahrheitstabelle 1-Bit Multiplexer}\label{tab:truthtable_1_bit_multiplexer}
\centering
\colorlet{BFH-table}{BFH-MediumBlue!10}
\colorlet{BFH-tablehead}{BFH-MediumBlue!50}
\setupBfhTabular
\begin{bfhTabular}{cccc}
In\_2 SelectSub & In\_1 SubIn   & In\_0 AddIn   & Y         \\\hline
\num{0}         & \num{0}       & \num{0}       & \num{0}   \\\hline
\num{0}         & \num{0}       & \num{1}       & \num{1}   \\\hline
\num{0}         & \num{1}       & \num{0}       & \num{0}   \\\hline
\num{0}         & \num{1}       & \num{1}       & \num{1}   \\\hline
\num{1}         & \num{0}       & \num{0}       & \num{0}   \\\hline
\num{1}         & \num{0}       & \num{1}       & \num{0}   \\\hline
\num{1}         & \num{1}       & \num{0}       & \num{1}   \\\hline
\num{1}         & \num{1}       & \num{1}       & \num{1}   \\\hline
\end{bfhTabular}
\end{table}
\subsubsection{\gls{kv}-Diagramme}
\begin{table}[H]
\caption{KV-Diagramm des 1-Bit Multiplexer}\label{tab:kv_multiplexer}
\centering
\colorlet{BFH-table}{BFH-MediumRed!10}
\colorlet{BFH-tablehead}{BFH-MediumRed!50}
\setupBfhTabular
\begin{bfhTabular}{lll}
Output & KV-Diagramm & konjunktive Normalform \\\hline
Y & \adjustbox{valign=t}{\includegraphics[width=0.3\textwidth]{multiplexer/kv_onebitmultiplexer.png}} & $Y=(In_0\cdot\overline{In_2})+(In_1\cdot In_2)$ \\\hline
\end{bfhTabular}
\end{table}
\subsubsection{Implementierung in Logisim-evolution}
\begin{figure}[H]
    \centering
    \includegraphics[width=1\textwidth]{multiplexer/logisim_onebitmultiplexer.png}
    \caption{Implementierung des 1-Bit Multiplexer in \gls{le}}
    \label{fig:logisim_onebitmultiplexer}
\end{figure}
\subsubsection{8-Bit-multiplexer}
\begin{figure}[H]
    \centering
    \includegraphics[width=1\textwidth]{multiplexer/logisim_eightbitmultiplexer.png}
    \caption{Implementierung des 8-Bit Multiplexer in \gls{le}}
    \label{fig:logisim_eightbitmultiplexer}
\end{figure}
\subsubsection{Test der Schaltungen}
\begin{figure}[H]
    \centering
    \includegraphics[width=1\textwidth]{multiplexer/logisim_test.png}
    \caption{Testen des 8-Bit Multiplexer in \gls{le}}
    \label{fig:logisim_test_eightbitmultiplexer}
\end{figure}
\begin{table}[H]
\caption{Testprotokoll 8-Bit Multiplexer}\label{tab:test_protocol_eightbitmultiplexer}
\centering
\colorlet{BFH-table}{BFH-MediumGreen!10}
\colorlet{BFH-tablehead}{BFH-MediumGreen!50}
\setupBfhTabular
\resizebox{\textwidth}{!}{
\begin{bfhTabular}{cc|cc|c|c|c|c}
\multicolumn{2}{c}{Input Xin}                                                           & \multicolumn{2}{c}{Input Yin}                                                             &SelectSub  & Erwarteter Wert       & Resultat Simulation   & Resultat Leguan-Board \\\hline
Digitalwert & Binärwert                                                                 & Digitalwert   & Binärwert                                                                 &           &                       &                       &                       \\\hline
\num{60}    & \num[minimum-integer-digits=4]{0011} \num[minimum-integer-digits=4]{1100} & \num{30}      & \num[minimum-integer-digits=4]{0001} \num[minimum-integer-digits=4]{1110} & \num{0}   & \num{90}              & \num{90}              & \num{90}              \\\hline
\num{15}    & \num[minimum-integer-digits=4]{0000} \num[minimum-integer-digits=4]{1111} & \num{16}      & \num[minimum-integer-digits=4]{0001} \num[minimum-integer-digits=4]{0000} & \num{0}   & \num{31}              & \num{31}              & \num{31}              \\\hline
\num{101}   & \num[minimum-integer-digits=4]{0110} \num[minimum-integer-digits=4]{0101} & \num{99}      & \num[minimum-integer-digits=4]{0110} \num[minimum-integer-digits=4]{0011} & \num{0}   & \num{200}             & \num{200}             & \num{200}             \\\hline
\num{255}   & \num[minimum-integer-digits=4]{1111} \num[minimum-integer-digits=4]{1111} & \num{255}     & \num[minimum-integer-digits=4]{1111} \num[minimum-integer-digits=4]{1111} & \num{0}   & overflow+\num{254}    & overflow+\num{254}    & overflow+\num{254}    \\\hline
\num{60}    & \num[minimum-integer-digits=4]{0011} \num[minimum-integer-digits=4]{1100} & \num{30}      & \num[minimum-integer-digits=4]{0001} \num[minimum-integer-digits=4]{1110} & \num{1}   & \num{60}              & \num{60}              & \num{60}              \\\hline
\num{15}    & \num[minimum-integer-digits=4]{0000} \num[minimum-integer-digits=4]{1111} & \num{16}      & \num[minimum-integer-digits=4]{0001} \num[minimum-integer-digits=4]{0000} & \num{1}   & underflow+\num{255}   & underflow+\num{255}   & underflow+\num{255}   \\\hline
\num{101}   & \num[minimum-integer-digits=4]{0110} \num[minimum-integer-digits=4]{0101} & \num{99}      & \num[minimum-integer-digits=4]{0110} \num[minimum-integer-digits=4]{0011} & \num{1}   & \num{2}               & \num{2}               & \num{2}               \\\hline
\num{255}   & \num[minimum-integer-digits=4]{1111} \num[minimum-integer-digits=4]{1111} & \num{255}     & \num[minimum-integer-digits=4]{1111} \num[minimum-integer-digits=4]{1111} & \num{1}   & \num{0}               & \num{0}               & \num{0}               \\\hline
\end{bfhTabular}
}
\end{table}

%----------------  Binär-zu-BCD-Konverter   --------------------------------------------------------------------------------------------------------------------------------------------------------
\subsection{Binär-zu-BCD-Konverter}
\label{sub:results_bin_to_bcd}
\subsubsection{Wahrheitstabelle}
\begin{table}[H]
\caption{Wahrheitstabelle Add3-Block}\label{tab:truthtable_add3}
\centering
\colorlet{BFH-table}{BFH-MediumBlue!10}
\colorlet{BFH-tablehead}{BFH-MediumBlue!50}
\setupBfhTabular
\begin{bfhTabular}{ccccccccc}
Input & In\_3 & In\_2 & In\_1 & In\_0 & Out\_3 & Out\_2 & Out\_1 & Out\_0 \\\hline
\num{0} & \num{0} & \num{0} & \num{0} & \num{0} & \num{0} & \num{0} & \num{0} & \num{0}\\\hline
\num{1} & \num{0} & \num{0} & \num{0} & \num{1} & \num{0} & \num{0} & \num{0} & \num{1}\\\hline
\num{2} & \num{0} & \num{0} & \num{1} & \num{0} & \num{0} & \num{0} & \num{1} & \num{0}\\\hline
\num{3} & \num{0} & \num{0} & \num{1} & \num{1} & \num{0} & \num{0} & \num{1} & \num{1}\\\hline
\num{4} & \num{0} & \num{1} & \num{0} & \num{0} & \num{0} & \num{1} & \num{0} & \num{0}\\\hline
\num{5} & \num{0} & \num{1} & \num{0} & \num{1} & \num{1} & \num{0} & \num{0} & \num{0}\\\hline
\num{6} & \num{0} & \num{1} & \num{1} & \num{0} & \num{1} & \num{0} & \num{0} & \num{1}\\\hline
\num{7} & \num{0} & \num{1} & \num{1} & \num{1} & \num{1} & \num{0} & \num{1} & \num{0}\\\hline
\num{8} & \num{1} & \num{0} & \num{0} & \num{0} & \num{1} & \num{0} & \num{1} & \num{1}\\\hline
\num{9} & \num{1} & \num{0} & \num{0} & \num{1} & \num{1} & \num{1} & \num{0} & \num{0}\\\hline
\num{10} & \num{1} & \num{0} & \num{1} & \num{0} & - & - & - & - \\\hline
\num{11} & \num{1} & \num{0} & \num{1} & \num{1} & - & - & - & - \\\hline
\num{12} & \num{1} & \num{1} & \num{0} & \num{0} & - & - & - & - \\\hline
\num{13} & \num{1} & \num{1} & \num{0} & \num{1} & - & - & - & - \\\hline
\num{14} & \num{1} & \num{1} & \num{1} & \num{0} & - & - & - & - \\\hline
\num{15} & \num{1} & \num{1} & \num{1} & \num{1} & - & - & - & - \\\hline
\end{bfhTabular}
\end{table}
\subsubsection{\gls{kv}-Diagramme}
\begin{table}[H]
\caption{KV-Diagramme und Normalform der Ausgänge des Add3-Blocks}\label{tab:kv_add3}
\centering
\colorlet{BFH-table}{BFH-MediumRed!10}
\colorlet{BFH-tablehead}{BFH-MediumRed!50}
\setupBfhTabular
\begin{bfhTabular}{lll}
Output & KV-Diagramm & Normalform \\\hline
Out\_0 LSB & \adjustbox{valign=t}{\includegraphics[width=0.3\textwidth]{bin_to_bcd/Out_0.png}} & $Out_0=(In_0\cdot\overline{In_2}\cdot\overline{In_3})+(In_3\cdot\overline{In_2}\cdot\overline{In_0})+(In_2\cdot\overline{In_0}\cdot In_1)$ \\\hline
Out\_1 & \adjustbox{valign=t}{\includegraphics[width=0.3\textwidth]{bin_to_bcd/Out_1.png}} & $Out_1=(\overline{In_0} + \overline{In_3})\cdot(In_0 + In_3)\cdot(\overline{In_2}+\overline{In_1}+ In_0)$ \\\hline
Out\_2 & \adjustbox{valign=t}{\includegraphics[width=0.3\textwidth]{bin_to_bcd/Out_2.png}} & $Out_2=(In_2\cdot\overline{In_1}\cdot \overline{In_0})+(In_3\cdot In_0)$ \\\hline
Out\_3 MSB & \adjustbox{valign=t}{\includegraphics[width=0.3\textwidth]{bin_to_bcd/Out_3.png}} & $Out_3=(In_2\cdot In_0)+(In_2\cdot In_1)+In_3$ \\\hline
\end{bfhTabular}
\end{table}
\subsubsection{Implementierung in Logisim-evolution}
\begin{figure}[H]
    \centering
    \includegraphics[width=1\textwidth]{bin_to_bcd/Logisim_Add3.png}
    \caption{Implementierung des Add3-Blocks in \gls{le}}
    \label{fig:logisim_add3}
\end{figure}
\subsubsection{Gesamtschaltung}
\begin{figure}[H]
    \centering
    \includegraphics[width=1\textwidth]{bin_to_bcd/Logisim_bin_to_bcd.png}
    \caption{Implementierung des selbst erstellten Binär zu BCD Wandlers in \gls{le}}
    \label{fig:logisim_bin_to_bcd}
\end{figure}
\subsubsection{Test der Schaltungen}
\begin{figure}[H]
    \centering
    \includegraphics[width=1\textwidth]{bin_to_bcd/Logisim_test.png}
    \caption{Testen des Umwandlers in \gls{le}}
    \label{fig:logisim_bin_to_bcd}
\end{figure}
\begin{table}[H]
\caption{Testprotokoll Binär zu BCD Umwandler}\label{tab:test_protocol_bin_to_bcd}
\centering
\colorlet{BFH-table}{BFH-MediumGreen!10}
\colorlet{BFH-tablehead}{BFH-MediumGreen!50}
\setupBfhTabular
\resizebox{\textwidth}{!}{
\begin{bfhTabular}{cc|c|c|c}
\multicolumn{2}{c}{Input}                                                               & Erwarteter Wert       & Resultat Simulation   & Resultat Leguan-Board \\\hline
Digitalwert & Binärwert                                                                 &                       &                       &                       \\\hline
\num{0}     & \num[minimum-integer-digits=4]{0000} \num[minimum-integer-digits=4]{0000} & \num{0}               & \num{0}               & \num{0}               \\\hline
\num{1}     & \num[minimum-integer-digits=4]{0000} \num[minimum-integer-digits=4]{0001} & \num{1}               & \num{1}               & \num{1}               \\\hline
\num{9}     & \num[minimum-integer-digits=4]{0000} \num[minimum-integer-digits=4]{1001} & \num{9}               & \num{9}               & \num{9}               \\\hline
\num{10}    & \num[minimum-integer-digits=4]{0000} \num[minimum-integer-digits=4]{1010} & \num{10}              & \num{10}              & \num{10}              \\\hline
\num{98}    & \num[minimum-integer-digits=4]{0110} \num[minimum-integer-digits=4]{0010} & \num{98}              & \num{98}              & \num{98}              \\\hline
\num{99}    & \num[minimum-integer-digits=4]{0110} \num[minimum-integer-digits=4]{0011} & \num{99}              & \num{99}              & \num{99}              \\\hline
\num{100}   & \num[minimum-integer-digits=4]{0110} \num[minimum-integer-digits=4]{0100} & \num{100}             & \num{100}             & \num{100}             \\\hline
\num{199}   & \num[minimum-integer-digits=4]{1100} \num[minimum-integer-digits=4]{0111} & \num{199}             & \num{199}             & \num{199}             \\\hline
\num{200}   & \num[minimum-integer-digits=4]{1100} \num[minimum-integer-digits=4]{1000} & \num{200}             & \num{200}             & \num{200}             \\\hline
\num{255}   & \num[minimum-integer-digits=4]{1111} \num[minimum-integer-digits=4]{1111} & \num{255}             & \num{255}             & \num{255}             \\\hline
\end{bfhTabular}
}
\footnotesize
\end{table}

%----------------  Discussion   --------------------------------------------
\clearpage
\section{Diskussion}
\label{sec:discussion}
Dies ist unser erstes Laborprojekt an der \gls{bfh}. Alle geforderten Funktionen wurden erreicht. Da wir das ganze Dokument in \gls{latex} geschrieben haben, haben wir viel darüber gelernt.
Der Aufbau des Dokuments ist uns zuerst nicht gut gelungen. Es viel uns schwer die Kapitel Methoden und Resultate zu unterscheiden. Daher bedurf es im Nachhinein einer zeitintensiven Korrektur.
Durch dieses Projekt haben wir auch unser Wissen im Bereich \gls{le} und \gls{lb} aufbauen, sowie erweitern können. Für das nächste Mal müssen wir viel früher schon stärker darauf achten,
was in welchen Teil des Dokuments kommt. Da wir jetzt aber schon erste Erfahrungen mit \gls{latex} gesammelt, sowie auch eine passende Struktur erarbeitet haben, wird das Erstellen des Berichts
beim nächsten Mal vermutlich schon viel flüssiger laufen.
Insgesamt war das Projekt sehr zeitintensiv, da auch die Erwartungshaltung für uns noch nicht ganz klar ist, und noch ein bischen herausgespürt werden muss.

%----------------  Declaration of authorship   -----------------------------
\clearpage
\section*{Selbstständigkeitserklärung}
\label{sec:declaration_of_authorship}
\addcontentsline{toc}{section}{Selbstständigkeitserklärung}
\markboth{Selbstständigkeitserklärung}{Selbstständigkeitserklärung}
Ich bestätige, dass ich die vorliegende Arbeit selbstständig und ohne Benutzung anderer als der im Literaturverzeichnis angegebenen Quellen und 
Hilfsmittel angefertigt habe. Sämtliche Textstellen, die nicht von mir stammen, sind als Zitate gekennzeichnet und mit dem genauen Hinweis auf
ihre Herkunft versehen. Ich bestätige weiterhin, dass ich bei der Erstellung dieser Arbeit durchgehend steuernd gearbeitet habe und von einer
KI erzeugte Texte bzw. Textfragmente nicht unreflektiert übernommen habe. 

\begin{table}[H]
\centering
\begin{tabular}{ccc}
29. Oktober 2025  & \includegraphics[width=0.15\textwidth]{sig_example.png} & \includegraphics[width=0.15\textwidth]{sig_simon_eisele.png} \vspace{-0.45cm} \\
                  & \rule{0.35\textwidth}{0.2pt} & \rule{0.35\textwidth}{0.2pt} \\
                  & J. Aebischer & S. Eisele \\
\end{tabular}
\label{tab:declaration_of_authorship}
\end{table}

%----------------  Bibliography   ------------------------------------------
\clearpage
\printbibliography

%----------------  List of figures   ---------------------------------------
\ifnum\totvalue{figures}>0
  \clearpage
  \listoffigures
\fi
 
%----------------  List of tables   ----------------------------------------
\ifnum\totvalue{tables}>0
  \clearpage
  \listoftables
\fi

%----------------  List of listings   --------------------------------------
%\clearpage
%\lstlistoflistings 

%----------------  Glossary and acronyms  ----------------------------------
\clearpage
\printglossary[type=main]
\printglossary[type=acronym]

%------------ Index ----------------------
\clearpage
\printindex

%------------ Appendix ----------------	
\appendix
\section{Anhang}
\subsection{Logisim-Evolution Schaltpläne}
\label{app:logisim}
Die folgenden Dateien sind als Download verfügbar:
\begin{itemize}
  \item Projekt 4: \attachfile{appendix/project4.circ}{project4.circ}
\end{itemize}

\subsection{PDF}
\label{app:pdf}
Die folgenden Dateien sind als Download verfügbar:
\begin{itemize}
  \item Konzept Binär-zu-BCD: \attachfile{appendix/concept_binary_to_bcd.pdf}{concept\_binary\_to\_bcd.pdf}
\end{itemize}

\end{document}
