Ziel dieses Laborprojekts ist es, die im Modul Digitaltechnik erlernten Grundlagen zu vertiefen und praktisch anzuwenden. Dazu wurden mehrere kombinatorische Schaltungen
entwickelt, deren logische Funktionen mithilfe der Booleschen Algebra und KV-Diagrammen vereinfacht wurden. Die entwickelten Schaltungen wurden in Logisim-evolution umgesetzt,
simuliert und anschliessend auf dem Leguan-Board der BFH getestet. Durch die Verknüpfung von Theorie, Simulation und praktischer Umsetzung konnte das Verständnis für den Aufbau
und die Funktionsweise digitaler Logiksysteme gezielt erweitert werden. Bei den Logikfunktionen handelt es sich um folgende Baugruppen. BCD-zu-7-Segment-Decoder, Ripple-Carry-Addierer,
 Borrow-Bit-Subtrahierer, Ergebnisumschalter und Binär-zu-BCD-Converter.