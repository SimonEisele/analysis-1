Ziel des Projekts 4 „Kombinatorische Logik“ war es, die Software \gls{le}~\cite{burchLogisimevolution2024}\footnote{\url{https://github.com/logisim-evolution/logisim-evolution}},
sowie das \gls{lb}~\cite{HomeLeguanDevelopers}\footnote{\url{https://leguan.ti.bfh.ch/}} der \gls{bfh} kennenzulernen. Dabei sollten die aus dem Modul BTE5021-Digital
\glqq Elektronik Grundlagen\grqq~gewonnenen Kenntnisse angewendet und gefestigt werden.

Hierfür wurden verschiedene Schaltungen entwickelt, in \gls{le} aufgebaut und zum Testen auf den \gls{fpga} des \gls{lb} geladen. Die zu entwickelnden Schaltungen wurden in der
Aufgabenstellung~\cite{tamselKombinatorischeLogikProjekt2025} definiert:
\begin{itemize}
    \item \gls{bcd}-zu-7-Segment-Decoder
    \item Ripple-Carry-Addierer
    \item Borrow-Bit-Subtrahierer
    \item Ergebnisumschalter
    \item Binär-zu-\gls{bcd}-Konverter
\end{itemize}

\noindent Für die Entwicklung der Schaltungen wurden jeweils Wahrheitstabellen erstellt. Daraus konnten die entsprechenden \gls{kv}-Diagramme abgeleitet und anschliessend die minimalen Funktionen in
disjunktiver (Minterme) oder konjunktiver (Maxterme) Form bestimmt werden. Beim \gls{bcd}-zu-7-Segment-Decoder mussten zusätzliche Bedingungen berücksichtigt werden. Für die Funktionen
durften jeweils nur bestimmte Grundgattertypen verwendet werden. Dies bot die Möglichkeit, das Umformen mittels der Booleschen Algebra anzuwenden und zu üben.

Vor der praktischen Durchführung des Versuchs wurden die Vorbereitungsaufgaben~\cite[p.~3]{tamselKombinatorischeLogikProjekt2025} bearbeitet. Dabei ging es darum, die Schaltungen
für den \gls{bcd}-zu-7-Segment-Decoder gemäss den gestellten Bedingungen zu entwickeln, sowie ein Konzept für die logische Schaltung zur Umwandlung einer 8 bit-Binärzahl in 3 \gls{bcd}-Ziffern
zu erstellen.

Der praktische Versuch wurde im Labor durchgeführt und dauerte vier Lektionen. Im Anschluss blieb noch eine Woche, um fehlende Schaltungen zu vervollständigen
und den technischen Bericht fertigzustellen.

Die Grundlagen zur Durchführung dieses Projekts bildet das Modul BTE5021-Digital \glqq Elektronik Grundlagen\grqq. Die wichtigsten Inhalte sind im Skript
Digitaltechnik~\cite{jacometDigitaltechnikGrundlagen2023} zusammengefasst.