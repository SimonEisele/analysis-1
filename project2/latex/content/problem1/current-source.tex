%---------------- Part 2 -----------------------------------------------------------------------------------------------
\subsection{Teil 2 - Leistungsanpassung einer Stromquelle}
\label{sub:part02}
Im zweiten Teil wird die maximale Leistung analysiert, die eine Stromquelle an einen Lastwiderstand abgeben kann
(Leistungsanpassung). Die Schaltung dazu ist in \autoref{fig:part02-current-source-schematic} dargestellt.
\begin{figure}[H]
    \centering
    \includegraphics[width=0.5\textwidth]{part_02/current-source-schematic.png}
    \caption{Schaltung der Leistungsanpassung (Stromquelle)}
    \label{fig:part02-current-source-schematic}
\end{figure}

Für die Berechnungen werden folgende Werte der Stromquelle angenommen:
\begin{align*}
    I_\mathrm{0} = \SI{3}{\ampere} \\
    R_\mathrm{i} = \SI{5}{\ohm}
\end{align*}

%---------------- Herleitung der Leistung ------------------------------------------------------------------------------
\subsubsection{Herleitung der Leistung}
\label{subsub:part02-derivation-of-the-power}
Die elektrische Leistung am Lastwiderstand $R_\mathrm{L}$ wird gemäss Schaltung aus
\autoref{fig:part02-current-source-schematic} wie folgt berechnet:
\begin{equation}
    P = U \cdot I
    \label{eq:part02-power1}
\end{equation}

Die Spannung $U$ und der Strom $I$ können in Abhängigkeit der Leerlaufspannung $U_\mathrm{0}$ und der beiden Widerstände
$R_\mathrm{i}$ und $R_\mathrm{L}$ dargestellt werden. Die entsprechenden Formeln ergeben sich aus den Formeln für den
Spannungsteiler, sowie dem ohmschen Gesetz:
\begin{align}
    U &= I_\mathrm{0}\frac{R_\mathrm{i} \cdot R_\mathrm{L}}{R_\mathrm{i} + R_\mathrm{L}}
    \label{eq:part02-working-point-voltage} \\
    I &= I_\mathrm{0}\frac{R_\mathrm{i}}{R_\mathrm{i} + R_\mathrm{L}}
    \label{eq:part02-working-point-current}
\end{align}

Ersetzt man in \autoref{eq:part02-power1} $U$ und $I$ durch die gefundenen Äquivalente, so erhält man die Formel für den
Leistungsverlauf von  $P$ in Abhängigkeit des Widerstands $R_\mathrm{L}$.
\begin{equation}
    P(R_\mathrm{L}) = I_\mathrm{0}^2 R_\mathrm{i}^2\frac{R_\mathrm{L}}
        {(R_\mathrm{i} + R_\mathrm{L})^2}
    \label{eq:part02-power2}
\end{equation}

%---------------- Maximale Leistung ------------------------------------------------------------------------------------
\subsubsection{Maximale Leistung}
\label{subusub:part02-max-power}
Zur Bestimmung der Extremstellen der Leistungsfunktion $P(R_\mathrm{L})$ wird die erste Ableitung gebildet und gleich
null gesetzt. Dazu wird die Quotientenregel angewendet.
\begin{align}
    f'(x) = c \cdot \frac{u'(x) \cdot v(x) - u(x) \cdot v'(x)}{v^2(x)}
\end{align}
\begin{align*}
    c &= I_\mathrm{0}^2 R_\mathrm{i}^2 \\
    u(x) &= R_\mathrm{L} \\
    u'(x) &= 1 \\
    v(x) &= (R_\mathrm{i} + R_\mathrm{L})^2 \\
    v'(x) &= 2R_\mathrm{i} + 2R_\mathrm{L}
\end{align*}

Daraus ergibt sich die erste Ableitung wie folgt:
\begin{equation}
    P'(R_\mathrm{L}) = I_\mathrm{0}^2 R_\mathrm{i}^2\frac{(R_\mathrm{i} + R_\mathrm{L})^2
        - R_\mathrm{L}(2R_\mathrm{i} + 2R_\mathrm{L})}{(R_\mathrm{i} + R_\mathrm{L})^4}
\end{equation}

Durch Ausmultiplizieren und Wegkürzen folgt:
\begin{equation}
    P'(R_\mathrm{L}) = I_\mathrm{0}^2 R_\mathrm{i}^2\frac{(R_\mathrm{i} - R_\mathrm{L})}
        {(R_\mathrm{i} + R_\mathrm{L})^3}
\end{equation}

Da der Nenner in jedem Fall positiv ist, muss der Zähler null sein. Daraus folgt folgende Beziehung für den
Lastwiderstand $R_\mathrm{L}$ bei maximaler Leistung:
\begin{align}
    (R_\mathrm{i} - R_\mathrm{L}) &= 0 \\
    R_\mathrm{i} &= R_\mathrm{L}
\end{align}

Zur Klassifikation der Extremstelle wird die zweite Ableitung $P''(R_\mathrm{L})$ betrachtet.
\begin{equation}
    P''(R_\mathrm{L}) = -\frac{2I_\mathrm{0}^2 R_\mathrm{i}^2}{(R_\mathrm{i} + R_\mathrm{L})^3} < 0
\end{equation}
Da die zweite Ableitung $P''(R_\mathrm{L})$ bei $R_\mathrm{i} = R_\mathrm{L}$ kleiner als $0$ ist, handelt es sich um
ein Maximum.

Durch Einsetzen von $R_\mathrm{i} = R_\mathrm{L}$ in \autoref{eq:part02-power2} lässt sich eine Formel für die
Bestimmung der maximalen Leistung $P_\mathrm{0}$ herleiten:
\begin{equation}
    P_\mathrm{0} = I_\mathrm{0}^2 \frac{R_\mathrm{i}}{4}
    \label{eq:part02-max-power}
\end{equation}

Setzen wir die gegebenen Werte ein, erhalten wir die folgende maximale Leistung der Stromquelle aus
\autoref{fig:part02-current-source-schematic}:
\begin{equation*}
    P_\mathrm{0} = (\SI{3}{\ampere})^2 \cdot \frac{\SI{5}{\ohm}}{4} = \underline{\underline{\SI{11.25}{\watt}}}
\end{equation*}

Die maximale Leistung $P_\mathrm{0}$ der Stromquelle entspricht also der Leistung der äquivalenten Spannungsquelle,
welche in \autoref{sub:part01} berechnet wurde.

%---------------- Normierte Leistung -----------------------------------------------------------------------------------
\subsubsection{Normierte Leistung}
\label{subsub:part02_normed-power}
Durch Kombination von \autoref{eq:part02-power2} und \autoref{eq:part02-max-power} ergibt sich folgende Beziehung für
das Verhältnis von aktueller Leistung $P$ zu maximaler Leistung $P_\mathrm{0}$:
\begin{align}
    \frac{P}{P_\mathrm{0}} &= \frac{I_\mathrm{0}^2\frac{R_\mathrm{i}^2
        \cdot R_\mathrm{L}}{(R_\mathrm{i} + R_\mathrm{L})^2}}{I_\mathrm{0}^2 \frac{R_\mathrm{i}}{4}} \\
    \frac{P}{P_\mathrm{0}} &= \frac{4R_\mathrm{i}R_\mathrm{L}}{(R_\mathrm{i} + R_\mathrm{L})^2}
\end{align}

Durch Teilen des Nenners und des Zählers auf der rechten Seite der Gleichung durch $R_\mathrm{i}^2$ erhalten wir:
\begin{equation}
    \frac{P}{P_\mathrm{0}} = \frac{\frac{4R_\mathrm{L}}{R_\mathrm{i}}}
        {\left(1 + \frac{R_\mathrm{L}}{R_\mathrm{i}}\right)^2}
\end{equation}

Wenn wir jetzt $\frac{P}{P_\mathrm{0}} = p$ und $\frac{R_\mathrm{L}}{R_\mathrm{i}} = r$ setzen, ergibt sich:
\begin{equation}
    p(r) = \frac{4r}{(1 + r)^2}
    \label{eq:current-source-power-resistor}
\end{equation}

Diese Formel gibt die normierte Leistung ($P$ im Verhältnis zu $P_\mathrm{0}$) in Bezug zum Verhältnis des
Lastwiderstands $R_\mathrm{L}$ zum Innenwiderstand $R_\mathrm{i}$ der Stromquelle an. Die Normierung erleichtert die
Analyse, da sie den Leistungsverlauf unabhängig von den konkreten Strom- und Widerstandswerten beschreibt. Der
entsprechende Graph ist in \autoref{fig:part02-current-source-power-resistor} dargestellt. Die Achsen entsprechen dabei
folgenden Beziehungen:
\begin{align*}
    x-Achse &= r = \frac{R_\mathrm{L}}{R_\mathrm{i}} \\
    y-Achse &= p = \frac{P}{P_\mathrm{0}}
\end{align*}
\begin{figure}[H]
    \centering
    \includegraphics[width=0.5\textwidth]{part_02/current-source-power-resistor.png}
    \caption{Normierte Leistung in Bezug zum Verhältnis der Widerstände (Stromquelle)}
    \label{fig:part02-current-source-power-resistor}
\end{figure}

Der logarithmisch dargestellte Graph ist symmetrisch um $r = 1$, also $R_\mathrm{i} = R_\mathrm{L}$. Die Leistung für
$R_\mathrm{L} = \frac{1}{2} R_\mathrm{i}$ entspricht genau der Leistung bei $R_\mathrm{L} = 2R_\mathrm{i}$. Reziproke
Änderungen des Lastwiderstands $R_\mathrm{L}$ vom Innenwiderstand $R_\mathrm{i}$ nach oben und unten führen zu
identischen relativen Leistungseinbussen.

Die Stromquelle verhält sich also bezüglich der Leistung in Bezug zu den Widerständen $R_\mathrm{i}$ und $R_\mathrm{L}$
genau gleich wie die äquivalente Spannungsquelle. (Vgl. \autoref{sub:part01})


%---------------- Normierter Strom -------------------------------------------------------------------------------------
\subsubsection{Normierter Strom}
\label{subsub:part02_current-normed}
Wir betrachten noch einmal \autoref{eq:part02-working-point-current} und teilen beide Seiten durch $I_\mathrm{0}$, sowie
auf der rechten Seite den Nenner und den Zähler durch $R_\mathrm{i}$. Somit folgt:
\begin{equation}
    \frac{I}{I_\mathrm{0}} = \frac{1}{1 + \frac{R_\mathrm{L}}{R_\mathrm{i}}}
\end{equation}

Setzt man $\frac{I}{I_\mathrm{0}} = i$ und erneut $\frac{R_\mathrm{L}}{R_\mathrm{i}} = r$ setzen, erhalten wir:
\begin{equation}
    i(r) = \frac{1}{1 + r}
\end{equation}

Wir erhalten also den normierten Strom ($I$ im Verhältnis zu $I_\mathrm{0}$) in Bezug zum Verhältnis des
Lastwiderstands $R_\mathrm{L}$ zum Innenwiderstand $R_\mathrm{i}$ der Stromquelle. Der entsprechende Graph ist in
\autoref{fig:part02-current-source-current-resistor} dargestellt.
\begin{figure}[H]
    \centering
    \includegraphics[width=0.5\textwidth]{part_02/current-source-current-resistor.png}
    \caption{Normierter Strom in Bezug zum Verhältnis der Widerstände}
    \label{fig:part02-current-source-current-resistor}
\end{figure}

Der logarithmisch dargestellte Graph ist punktsymmetrisch um $(r = 1, i = 0.5)$, also $R_\mathrm{i} = R_\mathrm{L}$. Die
Stromzunahme, wenn $R_\mathrm{L} = \frac{1}{2}R_\mathrm{i}$, entspricht exakt der Stromabnahme, wenn $R_\mathrm{L} =
2R_\mathrm{i}$. Wird der Lastwiderstand $R_\mathrm{L}$ gegenüber dem Innenwiderstand $R_\mathrm{i}$ um einen Faktor
vergrössert, bzw. um den reziproken Faktor verkleinert, so ändert sich der normierte Strom um denselben Betrag.

Wie in \autoref{subsub:part01_voltage-normed} ersichtlich, ergibt die Summe des normierten Stroms und der normierten
Spannung an jedem Punkt $1$. Nimmt die normierte Spannung ab, so nimmt der normierte Strom zu und umgekehrt. (Vgl.
\autoref{fig:part01-voltage-source-voltage-resistor} und \autoref{fig:part02-current-source-current-resistor})