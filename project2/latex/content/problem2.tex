%---------------- Exercise ---------------------------------------------------------------------------------------------
\subsection{Übung}
\label{sec:exercise}
Wir betrachten folgende Funktion:
\begin{equation}
    f(x) = cos(2x^3 + 1)e^{x-1}
\end{equation}
Eine Funktion $f$ lässt sich als Summe zweier Teilfunktionen $g$ und $h$ schreiben:
\begin{equation}
    f(x) = g(x) + h(x)
\end{equation}
wobei:
\begin{align}
    g(x) &= \frac{f(x) + f(-x)}{2} (gerade) \\
    h(x) &= \frac{f(x) - f(-x)}{2} (ungerade)
\end{align}
Wenden wir dies auf unsere Funktion $f$ an, so erhalten wir folgende zwei Teilfunktionen:
\begin{align}
    g(x) = \frac{e{-1}}{2}\left(\cos\left(2x^3 + 1\right)e^{x} + \cos\left(-2x^3 + 1\right)e^{-x}\right) \\
    h(x) = \frac{e^{-1}}{2}\left(\cos\left(2x^3 + 1\right)e^{x} - \cos\left(-2x^3 + 1\right)e^{-x}\right)
\end{align}

Die beiden Teilfunktionen sind in \autoref{fig:problem03-function-g} und \autoref{fig:problem03-function-h} ersichtlich.
\begin{figure}[H]
    \centering
    \begin{minipage}[b]{0.45\textwidth}
        \centering
        \includegraphics[width=\textwidth]{problem_03/function-g.png}
        \caption{Funktion $g(x)$ (gerade)}
        \label{fig:problem03-function-g}
    \end{minipage}
    \hfill
    \begin{minipage}[b]{0.45\textwidth}
        \centering \includegraphics[width=\textwidth]{problem_03/function-h.png}
        \caption{Funktion $h(x)$ (ungerade)}
        \label{fig:problem03-function-h}
    \end{minipage}
\end{figure}

Die Summe der beiden Teilfunktionen $g(x)$ und $h(x)$ ist in \autoref{fig:problem03-function-g-h} dargestellt. Rechts
daneben in \autoref{fig:problem03-function-f} ist die originale Funktion ersichtlich.
\begin{figure}[H]
    \centering
    \begin{minipage}[b]{0.45\textwidth}
        \centering
        \includegraphics[width=\textwidth]{problem_03/function-g-h.png}
        \caption{Funktion $g(x) + h(x)$}
        \label{fig:problem03-function-g-h}
    \end{minipage}
    \hfill
    \begin{minipage}[b]{0.45\textwidth}
        \centering \includegraphics[width=\textwidth]{problem_03/function-f.png}
        \caption{Originalfunktion $f(x)$}
        \label{fig:problem03-function-f}
    \end{minipage}
\end{figure}
Die Funktion $f(x)$ lässt sich eindeutig in eine gerade und eine ungerade Funktion zerlegen. Die grafische Darstellung
bestätigt, dass die Summe dieser beiden Teilfunktionen ($g(x)$ und $h(x)$) exakt der ursprünglichen Funktion $f(x)$
entspricht.

Damit die Addition der beiden Funktionen $g(x)$ und $h(x)$ zu $f(x)$ besser ersichtlich ist, wurden die drei Funktionen
zusammen auf einem Graphen in \autoref{fig:part03-function-f-g-h} dargestellt.
\begin{figure}[H]
    \centering
    \includegraphics[width=0.8\textwidth]{problem_03/function-f-g-h.png}
    \caption{Addition der Funktionen $g(x)$ und $h(x)$ zu $f(x)$}
    \label{fig:part03-function-f-g-h}
\end{figure}