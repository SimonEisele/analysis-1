In Übung 3 geht es um die Annäherung der Ableitung. Die Ableitung einer Funktion kann allgemein mit folgendem Grenzwert bestimmt werden:
\begin{equation*}
    f'(x) = \lim_{h \to 0} \frac{f(x + h) - f(x)}{h}
\end{equation*}
\noindent Mit dieser Formel nähert man sich dem Punkt der Ableitung von nur einer Seite an. Man rechnet also die Steigung der Funktion zwischen dem Punkt, an welchem
man die Ableitung ermitteln möchte und einem zweiten Punkt, welcher um $h$ verschoben ist. Um sich dem genauen Wert der Ableitung an einem Punkt noch schneller
zu nähern, kann man folgende Formel verwenden:
\begin{equation*}
    f'(x) = \lim_{h \to 0} \frac{f(x + \frac{h}{2}) - f(x - \frac{h}{2})}{h}
\end{equation*}
\noindent Hiermit rechnet man also die Steigung zwischen zwei Punkten, welche einmal $\frac{h}{2}$ vor dem Punkt der zu ermittelnden Ableitung um einmal $\frac{h}{2}$
dahinter liegen. Da hier beide Seiten des Punktes betrachtet werden, heben sich Fehler teilweise gegenseitig auf. Daher liefert diese Methode bereits bei grösserem $h$ eine
deutlich präzisere Annäherung.

%----------------  Derivation with different h   ---------------------------
\subsection{Annäherung der Ableitung von sin(x)}
Um dies an einem praktischen Beispiel zu untersuchen, versuchen wir die Ableitung von $\sin{x}$ durch die oben genannten Formeln zu bestimmen. Die korrekte Ableitung von
$\sin{x}$ ist wie folgt:
\begin{equation*}
    \sin'{x} = \cos{x}
\end{equation*}
Um den Unterschied und die Funktion der beiden Varianten der Annäherung zur Ableitung zu zeigen, wurden die verschiedenen Kurven im Intervall $[-\frac{\pi}{2}, \frac{\pi}{2}]$
mit jeweils immer kleiner werdendem $h$ geplottet. Auf den fünf Abbildungen ist deutlich ersichtlich, dass sich beide Varianten bei kleinem $h$ immer mehr an die korrekte 
Ableitung annähern, wobei aber die zweite Variante jeweils näher an der korrekten Ableitung liegt.
\begin{figure}[H]
    \centering
    \includegraphics[width=0.8\textwidth]{exercise03/exercise03-1.png}
    \caption{Annäherung der Ableitung einer Funktion - Schritt 1: h = 5}
    \label{fig:exercise03-1}
\end{figure}

\begin{figure}[H]
    \centering
    \includegraphics[width=0.8\textwidth]{exercise03/exercise03-2.png}
    \caption{Annäherung der Ableitung einer Funktion - Schritt 2: h = 3}
    \label{fig:exercise03-2}
\end{figure}

\begin{figure}[H]
    \centering
    \includegraphics[width=0.8\textwidth]{exercise03/exercise03-3.png}
    \caption{Annäherung der Ableitung einer Funktion - Schritt 3: h = 1}
    \label{fig:exercise03-3}
\end{figure}

\begin{figure}[H]
    \centering
    \includegraphics[width=0.8\textwidth]{exercise03/exercise03-4.png}
    \caption{Annäherung der Ableitung einer Funktion - Schritt 4: h = 0.5}
    \label{fig:exercise03-4}
\end{figure}

\begin{figure}[H]
    \centering
    \includegraphics[width=0.8\textwidth]{exercise03/exercise03-5.png}
    \caption{Annäherung der Ableitung einer Funktion - Schritt 5: h = 0.01}
    \label{fig:exercise03-5}
\end{figure}